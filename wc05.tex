\documentclass[a4paper]{exam}

\usepackage{geometry}
\usepackage{graphicx}
\usepackage{hyperref}
\usepackage{mathtools}
\usepackage{titling}

\printanswers

\title{Weekly Challenge 05: Master Theorem\\CS 412 Algorithms: Design and Analysis}
\author{q1-Team-05}  % <==== replace with your team name for grading
\date{Habib University | Spring 2023}

\runningheader{CS 412: Algorithms}{WC05: Master Theorem}{\theauthor}
\runningheadrule
\runningfootrule
\runningfooter{}{Page \thepage\ of \numpages}{}

\qformat{{\large\bf \thequestion. \thequestiontitle}\hfill}
\boxedpoints

\begin{document}
\maketitle

\begin{questions}

  
  \titledquestion{Reverse Engineering}

  We are going to design algorithms to meet a target asymptotic time complexity, and then investigate their running time.

  \subsection*{Tasks}
  \begin{description}
  \item[Theorem] State the master theorem.
  \item[Recurrences 1] Design distinct recurrences, $T_A(n)$ and $T_B(n)$, each of which solve to $\Theta(n^2)$.
  \item[Recurrences 2] Design distinct recurrences, $T_C(n)$ and $T_D(n)$, each of which solve to $\Theta(n\lg n)$.
  \item[Justify] Use the master theorem to justify the claimed solutions of  $T_A(n)$, $T_B(n)$, $T_C(n)$, and $T_D(n)$.
  \item[Solve 1] Use any other method to solve any one of  $T_A(n)$ and $T_B(n)$.
  \item[Solve 2] Use any other method to solve any one of  $T_C(n)$ and $T_D(n)$.
  \item[Implement 1] In the accompanying file, \texttt{algos.py}, implement plausible algorithms whose running times match $T_A(n),T_B(n), T_C(n)$, and $T_D(n)$. You may modify the parameter list of each as required.
  \item[Timing] Time the run time of your 4 algorithms In the accompanying file, \texttt{algos.py}, across a wide range of values of $n$.
  \item[Compare] Plot the run time of your algorithms in a single diagram. Make sure that each plot is clearly labeled, or the diagram contains a clearly visible legend. Make sure that the axis limits are set such that the plots are clearly visible and occupy a large portion of the diagram. Include separate diagrams over different axis ranges for greater clarity if necessary.
  \item[Submit] Include the diagram in your solution below.
  \item[Share] Share your diagram as a comment on the \href{https://web.yammer.com/main/org/habib.edu.pk/threads/eyJfdHlwZSI6IlRocmVhZCIsImlkIjoiMjEyMzI4NzkzMzY4MTY2NCJ9}{WC05 post} in the course group.
  \end{description}
  \underline{Tip}: You may consider \href{https://stackoverflow.com/a/68054319/1382487}{\texttt{time.perf\_counter()}} (detailed tutorial \href{https://realpython.com/python-timer/#other-python-timer-functions}{here}) for timing purposes.
  
  \begin{solution}
  \\
  Master Theorem:\\
  Theorem 4.1 (Master theorem)
  Let $a > 0$ and $b > 1$ be constants, and let f(n) be a driving function that is
  defined and non-negative on all sufficiently large reals. Define the recurrence T(n) on $n \in N$ by\\
  \\
  $T(n) = aT(n/b) + f(n)$\\
  
  Where $aT(n/b)$ actually means $a'T([n/b])+a''T([n/b])$ 
  \\for some constants $a'\geq 0$ and $a''\geq 0$ satisfying $a=a'+a''$. Then the asymptotic behavior of $T(n)$ can be characterized as follows: 
  \begin{enumerate}
  \item If there exists a constant $\epsilon > 0$ such that $f(n)=O(n^{\log_{b}a - \epsilon})$ then $T(n)=\theta(n^{\log_{b} a})$
  \item If there exists a constant $k > 0$ such that $f(n)=\theta(n^{\log_{b} a} lg^k n)$, then $T(n)=\theta(n^{\log_{b} a} lg^{k+1} n)$
  \item If there exists a constant $\epsilon > 0$ such that $f(n)= \Omega (n^{\log_{b}a + \epsilon})$ if f(n) additionally satisfies the regularity condition $af(n/b) \leq cf(n)$ for some constant
  $c < 1$ and all sufficiently large n, then $T(n) =
  \theta(f(n))$ 
  \end{enumerate}
  
  \end{solution}
  
  
  \begin{solution}
  \\
  \\\textbf{Recurrences 1:}
  \begin{itemize}
  \item $T_A(n) = 2 T_A(\frac{n}{2}) + n^2$
  \item $T_B(n) = 1 T_B(\frac{n}{2}) + n^2$
  \end{itemize}

  \textbf{Recurrences 2:}
  \begin{itemize}
  \item $T_C(n) = 2T_C(\frac{n}{2}) + n$
  \item $T_D(n) = 3T_D(\frac{n}{3}) + n$
  \end{itemize}
  
  \textbf{Justify:}\\
  \\
  $T_A(n)$:
  \\$a=2$
  \\$b=2$
  \\$d=2$
  \\case: $d>\log_b{a}$
  \\$2>\log_2{2}$
  \\Hence The time complexity is $T_A(n) = O(n^d) = O(n^2)$
  \\
  \\
  $T_B(n)$:
  \\$a=1$
  \\$b=2$
  \\$d=2$
  \\case: $d>\log_b{a}$
  \\$2>\log_2{1}$
  \\Hence The time complexity is $T_B(n) = O(n^d) = O(n^2)$
  \\
  \\
  $T_C(n)$:
  \\$a=2$
  \\$b=2$
  \\$d=1$
  \\case: $d=\log_b{a}$
  \\$1=\log_2{2}$
  \\Hence The time complexity is $T_C(n) = O(n^dlogn) = O(nlogn)$
  \\
  \\
  $T_D(n)$:
  \\$a=3$
  \\$b=3$
  \\$d=1$
  \\case: $d=\log_b{a}$ 
  \\$1=\log_3{3}$
  \\Hence The time complexity is $T_D(n) = O(n^dlogn) = O(nlogn)$
  \end{solution}

  \begin{solution}
  \textbf{Solve 1:}\\
  $T_A(n)$:
  \\Using Back substitution method:\\
  When $n=1$
  \\$T_A(n) = 2 T_A(\frac{n}{2}) + n^2$
  \\$T_A(n)=1$
  \\When $n>1$;
  \\
  \\First Equation:
  \\$T_A(n)= 2T_C(\frac{n}{2}) + n$
  \\Expanding for n/2;
  \\$T_A(n/2)= 2T_A(\frac{n}{2^2}) + n^2/2^2$
  \\Substituting back in First equation;
  \\$T_A(n)= 2[2T_A(\frac{n}{2^2}) + n^2/2^2] + n^2$
  \\
  \\Second equation:
  \\$T_A(n)= 2^2T_A(\frac{n}{2^2}) + n^2/2 + n^2$
  \\Expanding for $n/2^2$;
  \\$T_A(n/2^2)= 2T_A(\frac{n}{2^3}) + n^2/2^4$
  \\Substituting back in Second equation;
  \\$T_A(n)= 2^2[2T_A(\frac{n}{2^3}) + n^2/2^4] + n^2/2 + n^2$
  \\Third equation:
  \\$T_A(n)= 2^3T_A(\frac{n}{2^3}) + n^2/2^2 + n^2/2 + n^2$
  \\For k times:
  \\$T_A(n)= 2^kT_A(\frac{n}{2^k}) + n^2(1/2^2+1/2+1)$
  \\We assume $T_C(n/2^k) = T(1)$ 
  \\Then, $n/s^k = 1$
  \\$n = 2^k$
  \\$k=^n\log 2$
  \\
  \\$T_A(n)= 2^kT_A(1) + n^2$
  \\$T_A(n)= 2^n\log 2 + n^2$
  \\$O(n^2)$
\textbf{Solve 2:}\\
\\
$T_C(n)$:
\\
Using Back substitution method:\\
When $n=1$;
\\$T_C(n)=1$
\\When $n>1$;
\\$T_C(n)= 2T_C(\frac{n}{2}) + n$
\\
\\First equation:
\\$T_C(n)= 2T_C(\frac{n}{2}) + n$
\\Expanding for n/2;
\\$T_C(n/2)= 2T_C(\frac{n}{2^2}) + n/2$
\\Substituting back in First equation;
\\$T_C(n)= 2[2T_C(\frac{n}{2^2}) + n/2] + n$
\\
\\Second equation:
\\$T_C(n)= 2^2T_C(\frac{n}{2^2}) + 2n$
\\Expanding for $n/2^2$;
\\$T_C(n/2^2)= 2T_C(\frac{n}{2^3}) + n/2^2$
\\Substituting back in Second equation;
\\$T_C(n)= 2^2[2T_C(\frac{n}{2^3}) + n/2^2] + 2n$
\\
\\Third equation:
\\$T_C(n)= 2^3T_C(\frac{n}{2^3}) + 3n$
\\
\\For k times:
\\$T_C(n)= 2^kT_C(\frac{n}{2^k}) + kn$
\\
\\We assume $T_C(n/2^k) = T(1)$ 
\\Then, $n/s^k = 1$
\\$n = 2^k$
\\$logn = klog2$
\\$k=logn$
\\
\\Substituting the calculated values;
\\$2^k=1, T(1)=1, k=logn$
\\$T_C(n)= 2^kT_C(1) + kn$
\\$T_C(n)= n.1 + nlogn$
\\Hence, $T_C(n)=\Theta(nlogn)$
  \end{solution}
References:

  
\end{questions}

\end{document}  
